\documentclass{article}
\usepackage[utf8]{inputenc}
\usepackage[T1]{fontenc}
\usepackage{indentfirst, hyperref, gensymb, amsmath, amsthm, wasysym, amsfonts, mathtools, braket, amssymb}
\hypersetup{colorlinks,allcolors=blue,linktoc=all}

\let\biconditional\leftrightarrow
\let\conditional\rightarrow

\title{Chapter 1 Section 3 Exercise Solutions}
\author{Samuel Lair}
\date{September 2022}

\begin{document}

\maketitle
\tableofcontents

\pagebreak

\section{Exercise 5}
\[
	\begin{array}{|c|c|c|c|c|c|c|c|}
		p & q & r & q \lor r & p \land (q \lor r) & p \land q & p \land r & (p \land q) \lor (p \land r) \\
		\hline
		T & T & T & T        & T                  & T         & T         & T                            \\
		T & T & F & T        & T                  & T         & F         & T                            \\
		T & F & T & T        & T                  & F         & T         & T                            \\
		T & F & F & F        & F                  & F         & F         & F                            \\
		F & T & T & T        & F                  & F         & F         & F                            \\
		F & T & F & T        & F                  & F         & F         & F                            \\
		F & F & T & T        & F                  & F         & F         & F                            \\
		F & F & F & F        & F                  & F         & F         & F
	\end{array}
\]

Since the truth values of the compound propositions $p \land (q \lor r)$ and $(p \land q) \lor (p \land r)$ agree for all possible combinations of the truth values of $p$, $q$, and $r$, said compound propositions are logically equivalent.

\section{Exercise 6}
\[
	\begin{array}{|c|c|c|c|c|c|c|}
		p & q & p \land q & \neg (p \land q) & \neg p & \neg q & \neg p \lor \neg q \\
		\hline
		T & T & T         & F                & F      & F      & F                  \\
		T & F & F         & T                & F      & T      & T                  \\
		F & T & F         & T                & T      & F      & T                  \\
		F & F & F         & T                & T      & T      & T
	\end{array}
\]

Since the truth values of the compound propositions $\neg (p \land q)$ and $\neg p \lor \neg q$ agree for all possible combinations of the truth values of $p$ and $q$, said compound propositions are logically equivalent.

\pagebreak

\section{Exercise 9}

\subsection{(a)}
\begin{align*}
	     & p \implies \neg q & \equiv \\
	\neg & p \lor \neg q
\end{align*}

\subsection{(b)}
\begin{align*}
	     & (p \implies q) \implies r & \equiv \\
	\neg & (p \implies q) \lor r     & \equiv \\
	\neg & (\neg p \lor q) \lor r    & \equiv \\
	     & (p \land \neg q) \lor r
\end{align*}

\subsection{(c)}
\begin{align*}
	     & (\neg q \implies p) \implies (p \implies \neg q) & \equiv \\
	\neg & (\neg q \implies p) \lor (p \implies \neg q)     & \equiv \\
	\neg & (q \lor p) \lor (\neg p \lor \neg q)             & \equiv \\
	     & (\neg q \land \neg p) \lor (\neg p \lor \neg q)  & \equiv \\
	     & (\neg p \land \neg q) \lor (\neg p \lor \neg q)  & \equiv \\
	     & ((\neg p \land \neg q) \lor \neg p) \lor \neg q  & \equiv \\
	     & (\neg p \lor (\neg p \land \neg q)) \lor \neg q  & \equiv \\
	\neg & p \lor \neg q
\end{align*}

\pagebreak

\section{Exercise 10}

\subsection{(a)}
\begin{align*}
	\neg       & p \implies \neg q & \equiv \\
	\neg (\neg & p) \lor \neg q    & \equiv \\
	           & p \lor \neg q
\end{align*}

\subsection{(b)}
\begin{align*}
	       & p \lor q \implies \neg p     & \equiv \\
	\neg ( & p \lor q) \lor \neg p        & \equiv \\
	(\neg  & p \land \neg q) \lor \neg p  & \equiv \\
	\neg   & p \lor (\neg p \land \neg q) & \equiv \\
	\neg   & p
\end{align*}

\subsection{(c)}
\begin{align*}
	(          & p \implies \neg q) \implies (\neg p \implies q) & \equiv \\
	\neg (     & p \implies \neg q) \lor (\neg p \implies q)     & \equiv \\
	\neg (\neg & p \lor \neg q) \lor (p \lor q)                  & \equiv \\
	(          & p \land q) \lor (p \lor q)                      & \equiv \\
	((         & p \land q) \lor p) \lor q                       & \equiv \\
	(          & p \lor (p \land q)) \lor q                      & \equiv \\
	           & p \lor q
\end{align*}

\pagebreak

\section{Exercise 15}

\subsection{(a)}
\begin{align*}
	      & (p \land q) \implies p & \equiv \\
	\neg  & (p \land q) \lor p     & \equiv \\
	(\neg & p \lor \neg q) \lor p  & \equiv \\
	\neg  & q \lor (p \lor \neg p) & \equiv \\
	\neg  & q \lor T               & \equiv \\
	      & T
\end{align*}

\subsection{(b)}
\begin{align*}
	     & p \implies (p \lor q)  & \equiv \\
	\neg & p \lor (p \lor q)      & \equiv \\
	     & q \lor (p \lor \neg p) & \equiv \\
	     & q \lor T               & \equiv \\
	     & T
\end{align*}

\subsection{(c)}
\begin{align*}
	\neg & p \implies (p \implies q)  & \equiv \\
	\neg & p \implies (\neg p \lor q) & \equiv \\
	     & p \lor (\neg p \lor q)     & \equiv \\
	     & q \lor (p \lor \neg p)     & \equiv \\
	     & q \lor T                   & \equiv \\
	     & T
\end{align*}

\subsection{(d)}
\begin{align*}
	(      & p \land q) \implies (p \implies q)  & \equiv \\
	(      & p \land q) \implies (\neg p \lor q) & \equiv \\
	\neg ( & p \land q) \lor (\neg p \lor q)     & \equiv \\
	(\neg  & p \lor \neg q) \lor (\neg p \lor q) & \equiv \\
	(\neg  & p \lor \neg p) \lor (q \lor \neg q) & \equiv \\
	\neg   & p \lor T                            & \equiv \\
	       & T
\end{align*}

\subsection{(e)}
\begin{align*}
	\neg(     & p \implies q) \implies p & \equiv \\
	\neg(\neg & p \lor q) \implies p     & \equiv \\
	(\neg     & p \lor q) \lor p         & \equiv \\
	          & q \lor (p \lor \neg p)   & \equiv \\
	          & q \lor T                 & \equiv \\
	          & T
\end{align*}

\subsection{(f)}
\begin{align*}
	\neg(     & p \implies q) \implies \neg q & \equiv \\
	\neg(\neg & p \lor q) \implies \neg q     & \equiv \\
	(\neg     & p \lor q) \lor \neg q         & \equiv \\
	\neg      & p \lor (q \lor \neg q)        & \equiv \\
	\neg      & p \lor T                      & \equiv \\
	          & T
\end{align*}

\pagebreak

\section{Exercise 16}
\subsection{(a)}
\begin{align*}
	[\neg      & p \land (p \lor q)] \implies q                         & \equiv \\
	\neg [\neg & p \land (p \lor q)] \lor q                             & \equiv \\
	           & p \lor \neg (p \lor q) \lor q                          & \equiv \\
	           & p \lor (\neg p \land \neg q) \lor q                    & \equiv \\
	(          & p \lor q) \lor (\neg p \land \neg q)                   & \equiv \\
	((         & p \lor q) \lor \neg p) \land ((p \lor q) \lor \neg q)  & \equiv \\
	(          & q \lor (p \lor \neg p)) \land (p \lor (q \lor \neg q)) & \equiv \\
	(          & q \lor T) \land (p \lor T)                             & \equiv \\
	           & T \land T                                              & \equiv \\
	           & T
\end{align*}

\subsection{(b)}
\begin{align*}
	[(          & p \implies q) \land (q \implies r)] \implies (p \implies r)                         & \equiv \\
	\neg [(     & p \implies q) \land (q \implies r)] \lor (p \implies r)                             & \equiv \\
	\neg [(\neg & p \lor q) \land (\neg q \lor r)] \lor (\neg p \lor r)                               & \equiv \\
	[\neg(\neg  & p \lor q) \lor \neg (\neg q \lor r)] \lor (\neg p \lor r)                           & \equiv \\
	[(          & p \land \neg q) \lor (q \land \neg r)] \lor (\neg p \lor r)                         & \equiv \\
	[((         & p \land \neg q) \land q) \lor ((p \lor \neg q) \land \neg r)] \lor (\neg p \lor r)  & \equiv \\
	[(          & p \land (q \land \neg q)) \lor ((p \lor \neg q) \land \neg r)] \lor (\neg p \lor r) & \equiv \\
	[(          & p \land F) \lor (\neg r \land (p \lor \neg q))] \lor (\neg p \lor r)                & \equiv \\
	[           & F \lor ((\neg r \land p) \lor (\neg r \land \neg q))] \lor (\neg p \lor r)          & \equiv \\
	((\neg      & r \land p) \lor (\neg r \land \neg q)) \lor (\neg p \lor r)                         & \equiv \\
	((\neg      & p \lor r) \lor (\neg r \land p)) \lor (\neg r \land \neg q)                         & \equiv \\
	(((\neg     & p \lor r) \lor \neg r) \land ((\neg p \lor r) \lor p)) \lor (\neg r \land \neg q)   & \equiv \\
	((\neg      & p \lor T) \land (r \lor T)) \lor (\neg r \land \neg q)                              & \equiv \\
	(           & T \land T) \lor (\neg r \land \neg q)                                               & \equiv \\
	(\neg       & r \land \neg q) \lor T                                                              & \equiv \\
	            & T
\end{align*}

\subsection{(c)}
\begin{align*}
	[       & p \land (p \implies q)] \implies q                     & \equiv \\
	\neg [  & p \land (\neg p \lor q)] \lor q                        & \equiv \\
	[\neg   & p \lor \neg (\neg p \lor q)] \lor q                    & \equiv \\
	[\neg   & p \lor (p \land \neg q)] \lor q                        & \equiv \\
	(\neg   & p \lor q) \lor (p \land \neg q)                        & \equiv \\
	[((\neg & p \lor q) \lor p) \land ((\neg p \lor q) \lor \neg q)] & \equiv \\
	[(      & q \lor T) \land (\neg p \lor T)]                       & \equiv \\
	        & T \land T \equiv                                                \\
	        & T
\end{align*}

\subsection{(d)}
\begin{align*}
	[(      & p \lor q) \land (p \implies r) \land (q \implies r)] \implies r                & \equiv \\
	\neg [( & p \lor q) \land (\neg p \lor r) \land (\neg q \lor r)] \lor r                  & \equiv \\
	[\neg ( & p \lor q) \lor \neg (\neg p \lor r) \lor \neg (\neg q \lor r)] \lor r          & \equiv \\
	[(\neg  & p \land \neg q) \lor (p \land \neg r) \lor (q \land \neg r)] \lor r            & \equiv \\
	[(\neg  & p \land \neg q) \lor (p \land \neg r)] \lor [r \lor (q \land \neg r)]          & \equiv \\
	[(\neg  & p \land \neg q) \lor (p \land \neg r)] \lor [(r \lor q) \land (r \lor \neg r)] & \equiv \\
	[(\neg  & p \land \neg q) \lor (p \land \neg r)] \lor (r \lor q)                         & \equiv \\
	[(      & r \lor q) \lor (\neg p \land \neg q)] \lor (p \land \neg r)                    & \equiv \\
	[((     & r \lor q) \lor \neg p) \land ((r \lor q) \lor \neg q)] \lor (p \land \neg r)   & \equiv \\
	((      & r \lor q) \lor \neg p) \lor (p \land \neg r)                                   & \equiv \\
	(((     & r \lor q) \lor \neg p) \lor p) \land (((r \lor q) \lor \neg p) \lor \neg r)    & \equiv \\
	((      & r \lor q) \lor (p \lor \neg p)) \land ((\neg p \lor q) \lor (r \lor \neg r))   & \equiv \\
	((      & r \lor q) \lor T) \land ((\neg p \lor q) \lor T)                               & \equiv \\
	        & T \land T                                                                      & \equiv \\
	        & T
\end{align*}

\pagebreak

\section{Exercise 17}

\subsection{(a)}
\[
	\begin{array}{|c|c|c|c|}
		p          & q & p \land q & p \lor (p \land q) \\
		\hline
		\textbf{T} & T & T         & \textbf{T}         \\
		\textbf{T} & F & F         & \textbf{T}         \\
		\textbf{F} & T & F         & \textbf{F}         \\
		\textbf{F} & F & F         & \textbf{F}
	\end{array}
\]

Since the truth values of $p \lor (p \land q)$ and $p$ agree for all possible combinations of truth values for $p$ and $q$, $p \lor (p \land q)$ and $p$ are logically equivalent. I.e. $p \lor (p \land q) \equiv p$ is true.

\subsection{(b)}
\[
	\begin{array}{|c|c|c|c|}
		p          & q & p \lor q & p \land (p \lor q) \\
		\hline
		\textbf{T} & T & T        & \textbf{T}         \\
		\textbf{T} & F & T        & \textbf{T}         \\
		\textbf{F} & T & T        & \textbf{F}         \\
		\textbf{F} & F & F        & \textbf{F}
	\end{array}
\]

Since the truth values of $p \land (p \lor q)$ and $p$ agree for all possible combinations of truth values for $p$ and $q$, $p \land (p \lor q)$ and $p$ are logically equivalent. I.e. $p \land (p \lor q) \equiv p$ is true.

\pagebreak

\section{Exercise 18}

\begin{align*}
	(\neg      & p \land (p \conditional q)) \conditional \neg q & \equiv \\
	\neg (\neg & p \land (\neg p \lor q)) \lor \neg q            & \equiv \\
	(          & p \lor \neg (\neg p \lor q)) \lor \neg q        & \equiv \\
	(          & p \lor (p \land \neg q)) \lor \neg q            & \equiv \\
	           & p \lor \neg q                                   & \equiv \\
	\neg       & p \conditional \neg q
\end{align*}

Hence, $(\neg p \land (p \conditional q)) \conditional \neg q$ is not a tautology.

\pagebreak

\section{Exercise 19}

\begin{align*}
	(\neg      & q \land (p \conditional q)) \conditional \neg q & \equiv \\
	\neg (\neg & q \land (\neg p \lor q)) \lor \neg q            & \equiv \\
	(          & q \lor \neg (\neg p \lor q)) \lor \neg q        & \equiv \\
	(          & q \lor (p \land \neg q)) \lor \neg q            & \equiv \\
	(          & p \land \neg q) \lor (q \lor \neg q)            & \equiv \\
	(          & p \land \neg q) \lor T                          & \equiv \\
	           & T
\end{align*}

Hence, $(\neg q \land (p \conditional q)) \conditional \neg q$ is a tautology.

\pagebreak

\section{Exercise 20}
Let
\begin{align}
	  & p \biconditional q \label{ex20eq1}                    \\
	( & p \land q) \lor (\neg p \land \neg q) \label{ex20eq2}
\end{align}

\[
	\begin{array}{|c|c|c|c|c|c|}
		p & q & \eqref{ex20eq1} & p \land q & \neg p \land \neg q & \eqref{ex20eq2} \\
		\hline
		T & T & \textbf{T}      & T         & F                   & \textbf{T}      \\
		T & F & \textbf{F}      & F         & F                   & \textbf{F}      \\
		F & T & \textbf{F}      & F         & F                   & \textbf{F}      \\
		F & F & \textbf{T}      & F         & T                   & \textbf{T}
	\end{array}
\]

Since the truth values of \eqref{ex20eq1} and \eqref{ex20eq2} agree for all possible combinations of truth values for $p$ and $q$, \eqref{ex20eq1} and \eqref{ex20eq2} are logically equivalent.

\pagebreak

\section{Exercise 21}
Let
\begin{align}
	\neg ( & p \biconditional q) \label{ex21eq1}     \\
	       & p \biconditional \neg q \label{ex21eq2}
\end{align}

\[
	\begin{array}{|c|c|c|c|c|c|}
		p & q & p \biconditional q & \eqref{ex21eq1} & \neg q & \eqref{ex21eq2} \\
		\hline
		T & T & T                  & \textbf{F}      & F      & \textbf{F}      \\
		T & F & F                  & \textbf{T}      & T      & \textbf{T}      \\
		F & T & F                  & \textbf{T}      & F      & \textbf{T}      \\
		F & F & T                  & \textbf{F}      & T      & \textbf{F}
	\end{array}
\]

Since the truth values of \eqref{ex21eq1} and \eqref{ex21eq2} agree for all possible combinations of truth values for $p$ and $q$, \eqref{ex21eq1} and \eqref{ex21eq2} are logically equivalent.

\pagebreak

\section{Exercise 22}
Let
\begin{align}
	     & p \conditional q \label{ex22eq1}      \\
	\neg & q \conditional \neg p \label{ex22eq2}
\end{align}

\[
	\begin{array}{|c|c|c|c|c|c|}
		p & q & \eqref{ex22eq1} & \neg q & \neg p & \eqref{ex22eq2} \\
		\hline
		T & T & \textbf{T}      & F      & F      & \textbf{T}      \\
		T & F & \textbf{F}      & T      & F      & \textbf{F}      \\
		F & T & \textbf{T}      & F      & T      & \textbf{T}      \\
		F & F & \textbf{T}      & T      & T      & \textbf{T}
	\end{array}
\]

Since the truth values of \eqref{ex22eq1} and \eqref{ex22eq2} agree for all possible combinations of truth values for $p$ and $q$, \eqref{ex22eq1} and \eqref{ex22eq2} are logically equivalent.

\pagebreak
\section{Exercise 23}
Let
\begin{align}
	\neg & p \biconditional q \label{ex23eq1}      \\
	     & p \biconditional \neg q \label{ex23eq2}
\end{align}

\[
	\begin{array}{|c|c|c|c|c|c|}
		p & q & \neg p & \eqref{ex23eq1} & \neg q & \eqref{ex23eq2} \\
		\hline
		T & T & F      & F               & F      & F               \\
		T & F & F      & T               & T      & T               \\
		F & T & T      & T               & F      & T               \\
		F & F & T      & F               & T      & F
	\end{array}
\]

Since the truth values of \eqref{ex23eq1} and \eqref{ex23eq2} agree for all possible combinations of truth values for $p$ and $q$, \eqref{ex23eq1} and \eqref{ex23eq2} are logically equivalent.

\pagebreak
\section{Exercise 24}
$\neg (p \oplus q)$ is true when $p \oplus q$ is false, which means that $p$ and $q$ share the same truth value. This is exactly when $p \biconditional q$ is true. Hence, $\neg (p \oplus q)$ and $p \biconditional q$ are logically equivalent.

\pagebreak
\section{Exercise 25}
$\neg (p \biconditional q)$ is true when $p \biconditional q$ is false, which means that $p$ and $q$ have different truth values. This is exactly when $\neg p \biconditional q$ is true. Hence, $\neg (p \biconditional q)$ and $\neg p \biconditional q$ are logically equivalent.

\pagebreak
\section{Exercise 26}
$(p \conditional q) \land (p \conditional r)$ is true when both $(p \conditional q)$ and $(p \conditional r)$ are true, which means either $p = F$ or both $q = T$ and $r = T$. This is exactly when $p \conditional (q \land r)$ is true. Hence, $(p \conditional q) \land (p \conditional r)$ and $p \conditional (q \land r)$ are logically equivalent.

\pagebreak
\section{Exercise 27}
$(p \conditional r) \land (q \conditional r)$ is true when both $(p \conditional r)$ and $(q \conditional r)$ are true, which means either $r = T$ or both $p = F$ and $q = F$. This is exactly when $(p \lor q) \conditional r$ is true. Hence, $(p \conditional r) \land (q \conditional r)$ and $(p \lor q) \conditional r$ are logically equivalent.

\pagebreak
\section{Exercise 28}
$(p \conditional q) \lor (p \conditional r)$ is true when either $(p \conditional q)$ or $(p \conditional r)$ is true, which means either $p = F$, $q = T$, or $r = T$. This is exactly when $p \conditional (q \lor r)$ is true. Hence, $(p \conditional q) \lor (p \conditional r)$ and $p \conditional (q \lor r)$ are logically equivalent.

\pagebreak
\section{Exercise 29}
$(p \conditional r) \lor (q \conditional r)$ is true when either $(p \conditional r)$ or $(q \conditional r)$ is true, which means either $p = F$, $q = F$, or $r = T$. This is exactly when $(p \land q) \conditional r$ is true. Hence, $(p \conditional r) \lor (q \conditional r)$ and $(p \land q) \conditional r$ are logically equivalent.

\pagebreak
\section{Exercise 30}
$\neg p \conditional (q \conditional r)$ is true when either $\neg p$ is false or $(q \conditional r)$ is true, which means that either $p = T$, $q = F$, or $r = T$. This is exactly when $q \conditional (p \lor r)$ is true. Hence, $\neg p \conditional (q \conditional r)$ and $q \implies (p \lor r)$ are logically equivalent.

\pagebreak
\section{Exercise 31}
$p \biconditional q$ is true when $p$ and $q$ share the same truth value. This is exactly when $(p \conditional q) \land (q \conditional p)$ is true. Hence, $p \biconditional q$ and $(p \conditional q) \land (q \conditional p)$ are logically equivalent.

\pagebreak
\section{Exercise 32}
$p \biconditional q$ is true when $p$ and $q$ share the same truth value. This is exactly when $\neg p \biconditional \neg q$ is true. Hence, $p \biconditional q$ and $\neg p \biconditional \neg q$ are logically equivalent.

\pagebreak
\section{Exercise 34}
\begin{align*}
	(       & p \lor q) \land (\neg p \lor r) \conditional (q \lor r)                   & \equiv \\
	\neg [( & p \lor q) \land (\neg p \lor r)] \lor (q \lor r)                          & \equiv \\
	[\neg ( & p \lor q) \lor \neg (\neg p \lor r)] \lor (q \lor r)                      & \equiv \\
	[(\neg  & p \land \neg q) \lor (p \land \neg r)] \lor (q \lor r)                    & \equiv \\
	(\neg   & p \land \neg q) \lor [(q \lor r) \lor (p \land \neg r)]                   & \equiv \\
	(\neg   & p \land \neg q) \lor [((q \lor r) \lor p) \land ((q \lor r) \lor \neg r)] & \equiv \\
	(\neg   & p \land \neg q) \lor [((q \lor r) \lor p) \land T]                        & \equiv \\
	((      & q \lor r) \lor p) \lor (\neg p \land \neg q)                              & \equiv \\
	(((     & q \lor r) \lor p) \lor \neg p) \land (((q \lor r) \lor p) \lor \neg q)    & \equiv \\
	        & T \land T                                                                 & \equiv \\
	        & T
\end{align*}

\end{document}