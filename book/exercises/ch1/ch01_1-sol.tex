\documentclass{article}
\usepackage[utf8]{inputenc}
\usepackage[T1]{fontenc}
\usepackage{indentfirst, hyperref, gensymb, amsmath, amsthm, wasysym, amsfonts, mathtools, braket, amssymb}
\hypersetup{colorlinks,allcolors=blue,linktoc=all}

\title{Chapter 1 Section 1 Exercise Solutions}
\author{Samuel Lair}
\date{September 2022}

\begin{document}

\maketitle
\tableofcontents

\pagebreak

\section{Exercise 24}
\subsection{(a)}
If you got promoted, then you washed the boss's car.
\subsection{(b)}
If there are winds from the south, then there will be a spring thaw.
\subsection{(c)}
If you bought the computer less than a year ago, then the warranty is good.
\subsection{(d)}
If Willy cheats, then he gets caught.
\subsection{(e)}
If you can access the website, then you paid a subscription fee.
\subsection{(f)}
If you know the right people, then you'll get elected.
\subsection{(g)}
If Carol is on a boat, then she gets seasick.

\pagebreak

\section{Exercise 25}
\subsection{(a)}
If the wind blows from the northeast, then it snows.
\subsection{(b)}
If it stays warm for a week, then the apple trees will bloom.
\subsection{(c)}
If the Pistons win the championship, then they beat the Lakers.
\subsection{(d)}
If you got to the top of Long's Peak, then you walked eight miles.
\subsection{(e)}
If you are world famous, then you will get tenure as a professor.
\subsection{(f)}
If you drive more than 400 miles, then you will need to buy gasoline.
\subsection{(g)}
If your guarantee is good, then you bought your CD player less than 90 days ago.
\subsection{(h)}
If the water isn't too cold, then Jan will go swimming.
\subsection{(i)}
If people believe in science, then we will have a future.

\pagebreak

\section{Exercise 26}
\subsection{(a)}
If I remembered to send you the address, then you sent me an e-mail message.
\subsection{(b)}
If you were born in the United States, then you are a citizen of this country.
\subsection{(c)}
If you keep your textbook, then it will be a useful reference in your future courses.
\subsection{(d)}
If their goalie plays well, then the Red Wings will win the Stanley Cup.
\subsection{(e)}
If you get the job, then you had the best credentials.
\subsection{(f)}
If there is a storm, then the beach erodes.
\subsection{(g)}
If you log on to the server, then you have a valid password.
\subsection{(h)}
If you don't begin your climb too late, then you will reach the summit.
\subsection{(i)}
If you are among the first 100 customers tomorrow, then you will get a free ice cream cone.

\pagebreak

\section{Exercise 27}
\subsection{(a)}
It is hot outside if and only if you buy an ice cream cone.
\subsection{(b)}
You win the contest if and only if you have the only winning ticket.
\subsection{(c)}
You get promoted if and only if you have connections.
\subsection{(d)}
You watch television if and only if your mind will decay.
\subsection{(e)}
The trains run late if and only if it is a day when I take the train.

\pagebreak

\section{Exercise 28}
\subsection{(a)}
You get an A in this course if and only if you learn how to solve discrete mathematics problems.
\subsection{(b)}
You will be informed if and only if you read the newspaper every day.
\subsection{(c)}
It rains if and only if it is a weekend day.
\subsection{(d)}
The wizard is not in if and only if you can see him.
\subsection{(e)}
My airplane flight is late if and only if I have to catch a connecting flight.

\pagebreak

\section{Exercise 29}
\subsection{(a)}
Original: If it snows today, then I will ski tomorrow.

Converse: If I will ski tomorrow, then it snows today.

Contrapositive: If I won't ski tomorrow, then it doesn't snow today.

Inverse: If it doesn't snow today, then I won't ski tomorrow.

\subsection{(b)}
Original: If there is going to be a quiz, then I come to class.

Converse: If I come to class, then there is going to be a quiz.

Contrapositive: If I don't come to class, then there isn't going to be a quiz.

Inverse: If there isn't going to be a quiz, then I don't come to class.

\subsection{(c)}
Original: If a positive integer is prime, then it has no divisors other than 1 and itself.

Converse: If a positive integer has no divisors other than 1 and itself, then it is prime.

Contrapositive: If a positive integer has divisors other than 1 and itself, then it isn't prime.

Inverse: If a positive integer isn't prime, then it has divisors other than 1 and itself.

\pagebreak

\section{Exercise 30}

\subsection{(a)}
Original: If it snows tonight, then I will stay at home.

Converse: If I will stay at home, then it snows tonight.

Contrapositive: If I won't stay at home, then it doesn't snow tonight.

Inverse: If it doesn't snow tonight, then I won't stay at home.

\subsection{(b)}
Original: If it is a sunny summer day, then I go to the beach.

Converse: If I go to the beach, then it is a sunny summer day.

Contrapositive: If I don't go the beach, then it isn't a sunny summer day.

Inverse: If it isn't a sunny summer day, then I don't go to the beach.

\subsection{(c)}
Original: If I stay up late, then I sleep until noon.

Converse: If I sleep until noon, then I stay up late.

Contrapositive: If I don't sleep until noon, then I don't stay up late.

Inverse: If I don't stay up late, then I don't sleep until noon.

\pagebreak

\section{Exercise 42}
Let
\begin{align}
	 & (p \lor \neg q) \label{e42e1}                                       \\
	 & (q \lor \neg r) \label{e42e2}                                       \\
	 & (r \lor \neg p) \label{e42e3}                                       \\
	 & \eqref{e42e1} \land \eqref{e42e2} \land \eqref{e42e3} \label{e42e4}
\end{align}

\eqref{e42e1} is true if and only if p is true or q is false.

\eqref{e42e2} is true if and only if q is true or r is false.

\eqref{e42e3} is true if and only if r is true or p is false.

\eqref{e42e4} is true if and only if \eqref{e42e1}, \eqref{e42e2}, and \eqref{e42e3} are all true.

Therefore, when $p,q,r$ have the same truth value, \eqref{e42e1}, \eqref{e42e2}, \eqref{e42e3}, and \eqref{e42e4} are all true. However, when one of $p,q,r$ has a different truth value from the other two, two of \eqref{e42e1}, \eqref{e42e2}, and \eqref{e42e3} are true while the other is false. \eqref{e42e4} is false in this case.

Hence, \eqref{e42e4} is true when $p$, $q$, and $r$ have the same truth value and false otherwise.

\pagebreak

\section{Exercise 43}
Let
\begin{align}
	 & (p \lor q \lor r) \label{e43e1}                 \\
	 & (\neg p \lor \neg q \lor \neg r) \label{e43e2}  \\
	 & \eqref{e43e1} \land \eqref{e43e2} \label{e43e3}
\end{align}

\eqref{e43e1} is true if and only if at least one of $p, q, r$ is true.

\eqref{e43e2} is true if and only if at least one of $p, q, r$ is false.

Hence, \eqref{e43e3} is true if and only if at least one of $p, q, r$ is true and at least one of $p, q, r$ is false.  I.e. \eqref{e43e3} is true if at least one of $p, q, r$ is true and at least one is false, but is false when all three variables have the same truth value


\pagebreak

\section{Exercise 44}
If $p_1, p_2, ..., p_n$ are $n$ propositions explain why
\begin{equation}\label{e44claim}
	\bigwedge_{i=1}^{n-1} \bigwedge_{j=i+1}^n (\neg p_i \lor \neg p_j)
\end{equation}
is true if and only if at most of $p_1, p_2, ..., p_n$ is true.

\eqref{e44claim} is a conjunction of disjuncts of the form
\begin{equation}\label{e44dj}
	\neg p_i \lor \neg p_j
\end{equation}
Due to the limits of the conjunctions, there is exactly one disjunct for every pair $(i,j)$ such that $1 \le i < j \le n$.  Therefore, if more than one of the $p_k$'s are true, then there exists a pair $(i',j')$ such that both $p_{i'}$ and $p_{j'}$ are true and the corresponding disjunct \eqref{e44dj} is false.  A single false disjunct is enough to cause \eqref{e44claim} to evaluate to false.

If no more than 1 of the $p_k$'s is true, then \eqref{e44dj} is true.  If \eqref{e44dj} is true, then no more 1 of the $p_k$'s is true. Our claim follows.

\pagebreak

\section{Exercise 45}
\begin{equation}
	\bigvee_{h=1}^n p_h
\end{equation}
is true if and only if at least one of the $p_k$'s are true.  Therefore,
\begin{equation}
	\left( \bigwedge_{i=1}^{n-1} \bigwedge_{j=i+1}^n (\neg p_i \lor \neg p_j) \right) \land \left( \bigvee_{h=1}^n p_h \right)
\end{equation}
is true if and only if exactly one of the $p_k$'s is true.

\pagebreak

\section{Exercise 47}
\subsection{(a)}
\begin{align*}
	 & \text{101 1110}             &                      \\
	 & \underline{\text{010 0001}} &                      \\
	 & \text{111 1111}             & (\text{bitwise OR})  \\
	 & \text{000 0000}             & (\text{bitwise AND}) \\
	 & \text{111 1111}             & (\text{bitwise XOR}) \\
\end{align*}

\subsection{(b)}
\begin{align*}
	 & \text{1111 0000}             &                      \\
	 & \underline{\text{1010 1010}} &                      \\
	 & \text{1111 1010}             & (\text{bitwise OR})  \\
	 & \text{1010 0000}             & (\text{bitwise AND}) \\
	 & \text{0101 1010}             & (\text{bitwise XOR}) \\
\end{align*}

\subsection{(c)}
\begin{align*}
	 & \text{00 0111 0001}             &                      \\
	 & \underline{\text{10 0100 1000}} &                      \\
	 & \text{10 0111 1001}             & (\text{bitwise OR})  \\
	 & \text{00 0100 0000}             & (\text{bitwise AND}) \\
	 & \text{10 0011 1001}             & (\text{bitwise XOR}) \\
\end{align*}

\subsection{(d)}
\begin{align*}
	 & \text{11 1111 1111}             &                      \\
	 & \underline{\text{00 0000 0000}} &                      \\
	 & \text{11 1111 1111}             & (\text{bitwise OR})  \\
	 & \text{00 0000 0000}             & (\text{bitwise AND}) \\
	 & \text{11 1111 1111}             & (\text{bitwise XOR}) \\
\end{align*}

\section{Exercise 48}
\subsection{(a)}
\begin{align*}
	         & \text{0 1011}                             \\
	\lor \;  & \underline{\text{1 1011}}                 \\
	         & \text{1 1011}                             \\
	         &                                           \\
	         & \text{1 1000}                             \\
	\land \; & \underline{\text{1 1011}}                 \\
	         & \text{1 1000} \; \leftarrow \text{answer}
\end{align*}

\subsection{(b)}
\begin{align*}
	         & \text{0 1111}                             \\
	\land \; & \underline{\text{1 0101}}                 \\
	         & \text{0 0101}                             \\
	         &                                           \\
	         & \text{0 0101}                             \\
	\lor \;  & \underline{\text{0 1000}}                 \\
	         & \text{0 1101} \; \leftarrow \text{answer}
\end{align*}

\subsection{(c)}
\begin{align*}
	          & \text{0 1010}                             \\
	\oplus \; & \underline{\text{1 1011}}                 \\
	          & \text{1 0001}                             \\
	          &                                           \\
	          & \text{1 0001}                             \\
	\oplus \; & \underline{\text{0 1000}}                 \\
	          & \text{1 1001} \; \leftarrow \text{answer}
\end{align*}

\subsection{(d)}
\begin{align*}
	         & \text{1 1011}                             \\
	\lor \;  & \underline{\text{0 1010}}                 \\
	         & \text{1 1011}                             \\
	         &                                           \\
	         & \text{1 0001}                             \\
	\lor \;  & \underline{\text{1 1011}}                 \\
	         & \text{1 1011}                             \\
	         &                                           \\
	         & \text{1 1011}                             \\
	\land \; & \underline{\text{1 1011}}                 \\
	         & \text{1 1011} \; \leftarrow \text{answer}
\end{align*}

\pagebreak

\section{Exercise 49}
The truth value of "Fred is not happy" is $1 - 0.8 = 0.2$.

The truth value of "John is not happy" is $1 - 0.4 = 0.6$.

\pagebreak

\section{Exercise 50}
The truth value of "Fred and John are happy" is $min(0.8, 0.4) = 0.4$.

"Neither Fred nor John is happy" is equivalent to "Fred is not happy and John is not happy". Therefore, the truth value of "Neither Fred nor John is happy" is $min(0.2, 0.6) = 0.2$.

\pagebreak

\section{Exercise 51}
The truth value of "Fred is happy or John is happy" is $max(0.8, 0.4) = 0.8$.

The truth value of "Fred is not happy or John is not happy" is $max(0.2, 0.6) = 0.6$.

\pagebreak

\section{Exercise 52}
"This statement is false" is not a proposition.  It is a declarative sentence but it doesn't have a definite truth value.  It is, in fact, a logical paradox. Regardless of whether you assume it is true or false, you reach a contradiction.

\pagebreak

\section{Exercise 53}
\subsection{(a)}
If we assume that the 100th statement is true, we reach a contradiction since we must conclude that the 100th statement is false. No contradiction is reached if the 100th statement is false. Therefore, the 100th statement is false, which implies that at least one of the statements are true.

If we take $1 \le n \le 98$ and assume that the $n^{th}$ statement is true, we reach a contradiction. If the $n^{th}$ statement is true, then $100 - n$ statements must true. Since $100 - n > 1$, there must be true statements other than $n^{th}$ statement. However, if more than 1 statement is true, they will contradict each other. No contradiction is reached if we assume that the $n^{th}$ statement is false. Therefore, statements $1-98$ are be false.

Now if we assume that the $99^{th}$ statement is false, we reach a contradiction in that all statements are false yet exactly 1 must be true. However, no contradiction is reached if we assume that the $99^{th}$ statement is true. Therefore, the 99th statement must be true.

Hence, we conclude that the $99^{th}$ statement is true and the rest are false.

\subsection{(b)}
First of all, note that if the $n^{th}$ statement is true then all prior statements are true as well.

If we assume that the $51^{st}$ statement is true, we reach a contradiction. If the $51^{st}$ statement is true, then statements $1-50$ must also be true. This leaves only statements $52-100$ available to be false. However, this is only a total of $49$ statements, contradicting the $51^{st}$ statement. Therefore, $51^{st}$ statement is false, which implies that statements $52-100$ are also false.

If we assume that the $50^{th}$ statement is false, then we reach a contradiction where at least $50$ statements are false since statements $51-100$ must also be false.  We reach no contradiction if we assume that the $50^{th}$ statement is true. Therefore, the $50^{th}$ statement is true, which implies that statements $1-49$ are also true.

Hence, we conclude that statements $1-50$ are true and statements $51-100$ are false.

\subsection{(c)}
If we assume that the $50^{th}$ statement is true, then we reach a contraction since statements $1-49$ must also be true leaving only statements $51-99$ available to be false.

However, if we assume that the $50^{th}$ statement is false, then we reach a contraction where at $50$ statements are false since statements $51-99$ must also be false.

Hence, we conclude that this system of statements represents an unsolvable logical paradox.

\end{document}