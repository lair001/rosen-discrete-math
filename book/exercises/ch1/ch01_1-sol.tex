\documentclass{article}
\usepackage[utf8]{inputenc}
\usepackage[T1]{fontenc}
\usepackage{indentfirst, hyperref, gensymb, amsmath, amsthm, wasysym, amsfonts, mathtools, braket, amssymb}
\hypersetup{colorlinks,allcolors=blue,linktoc=all}

\title{Chapter 1 Exercise Solutions}
\author{Samuel Lair}
\date{September 2022}

\begin{document}
	
	\maketitle
	\tableofcontents
	
	\pagebreak
	
	\section{Exercise 24}
	\subsection{(a)}
	If you got promoted, then you washed the boss's car.
	\subsection{(b)}
	If there are winds from the south, then there will be a spring thaw.
	\subsection{(c)}
	If you bought the computer less than a year ago, then the warranty is good.
	\subsection{(d)}
	If Willy cheats, then he gets caught.
	\subsection{(e)}
	If you can access the website, then you paid a subscription fee.
	\subsection{(f)}
	If you know the right people, then you'll get elected.
	\subsection{(g)}
	If Carol is on a boat, then she gets seasick.
	
	\pagebreak
	
	\section{Exercise 25}
	\subsection{(a)}
	If the wind blows from the northeast, then it snows.
	\subsection{(b)}
	If it stays warm for a week, then the apple trees will bloom.
	\subsection{(c)}
	If the Pistons win the championship, then they beat the Lakers.
	\subsection{(d)}
	If you got to the top of Long's Peak, then you walked eight miles.
	\subsection{(e)}
	If you are world famous, then you will get tenure as a professor.
	\subsection{(f)}
	If you drive more than 400 miles, then you will need to buy gasoline.
	\subsection{(g)}
	If your guarantee is good, then you bought your CD player less than 90 days ago.
	\subsection{(h)}
	If the water isn't too cold, then Jan will go swimming.
	\subsection{(i)}
	If people believe in science, then we will have a future.
	
	\pagebreak
	
	\section{Exercise 26}
	\subsection{(a)}
	If I remembered to send you the address, then you sent me an e-mail message.
	\subsection{(b)}
	If you were born in the United States, then you are a citizen of this country.
	\subsection{(c)}
	If you keep your textbook, then it will be a useful reference in your future courses.
	\subsection{(d)}
	If their goalie plays well, then the Red Wings will win the Stanley Cup.
	\subsection{(e)}
	If you get the job, then you had the best credentials.
	\subsection{(f)}
	If there is a storm, then the beach erodes.
	\subsection{(g)}
	If you log on to the server, then you have a valid password.
	\subsection{(h)}
	If you don't begin your climb too late, then you will reach the summit.
	\subsection{(i)}
	If you are among the first 100 customers tomorrow, then you will get a free ice cream cone.
	
	\pagebreak
	
	\section{Exercise 27}
	\subsection{(a)}
	It is hot outside if and only if you buy an ice cream cone.
	\subsection{(b)}
	You win the contest if and only if you have the only winning ticket.
	\subsection{(c)}
	You get promoted if and only if you have connections.
	\subsection{(d)}
	You watch television if and only if your mind will decay.
	\subsection{(e)}
	The trains run late if and only if it is a day when I take the train.
	
	\pagebreak
	
	\section{Exercise 28}
	\subsection{(a)}
	You get an A in this course if and only if you learn how to solve discrete mathematics problems.
	\subsection{(b)}
	You will be informed if and only if you read the newspaper every day.
	\subsection{(c)}
	It rains if and only if it is a weekend day.
	\subsection{(d)}
	The wizard is not in if and only if you can see him.
	\subsection{(e)}
	My airplane flight is late if and only if I have to catch a connecting flight.
	
	\pagebreak
	
	\section{Exercise 47}
	\subsection{(a)}
	\begin{align*}
		&\text{101 1110} & \\
		&\underline{\text{010 0001}} & \\
		&\text{111 1111} & (\text{bitwise OR}) \\
		&\text{000 0000} & (\text{bitwise AND}) \\
		&\text{111 1111} & (\text{bitwise XOR}) \\
	\end{align*}

	\subsection{(b)}
\begin{align*}
	&\text{1111 0000} & \\
	&\underline{\text{1010 1010}} & \\
	&\text{1111 1010} & (\text{bitwise OR}) \\
	&\text{1010 0000} & (\text{bitwise AND}) \\
	&\text{0101 1010} & (\text{bitwise XOR}) \\
\end{align*}
	
		\subsection{(c)}
	\begin{align*}
		&\text{00 0111 0001} & \\
		&\underline{\text{10 0100 1000}} & \\
		&\text{10 0111 1001} & (\text{bitwise OR}) \\
		&\text{00 0100 0000} & (\text{bitwise AND}) \\
		&\text{10 0011 1001} & (\text{bitwise XOR}) \\
	\end{align*}

		\subsection{(d)}
\begin{align*}
	&\text{11 1111 1111} & \\
	&\underline{\text{00 0000 0000}} & \\
	&\text{11 1111 1111} & (\text{bitwise OR}) \\
	&\text{00 0000 0000} & (\text{bitwise AND}) \\
	&\text{11 1111 1111} & (\text{bitwise XOR}) \\
\end{align*}

\section{Exercise 48}
\subsection{(a)}
\begin{align*}
	&\text{0 1011} \\
	\lor \; &\underline{\text{1 1011}} \\
	&\text{1 1011} \\
	& \\
	&\text{1 1000} \\
	\land \; &\underline{\text{1 1011}} \\
	&\text{1 1000} \; \leftarrow \text{answer}
\end{align*}

\subsection{(b)}
\begin{align*}
	&\text{0 1111} \\
	\land \; &\underline{\text{1 0101}} \\
	&\text{0 0101} \\
	& \\
	&\text{0 0101} \\
	\lor \; &\underline{\text{0 1000}} \\
	&\text{0 1101} \; \leftarrow \text{answer}
\end{align*}

\subsection{(c)}
\begin{align*}
	&\text{0 1010} \\
	\oplus \; &\underline{\text{1 1011}} \\
	&\text{1 0001} \\
	& \\
	&\text{1 0001} \\
	\oplus \; &\underline{\text{0 1000}} \\
	&\text{1 1001} \; \leftarrow \text{answer}
\end{align*}

\subsection{(d)}
\begin{align*}
	&\text{1 1011} \\
	\lor \; &\underline{\text{0 1010}} \\
	&\text{1 1011} \\
	& \\
	&\text{1 0001} \\
	\lor \; &\underline{\text{1 1011}} \\
	&\text{1 1011} \\
	& \\
	&\text{1 1011} \\
	\land \; &\underline{\text{1 1011}} \\
	&\text{1 1011} \; \leftarrow \text{answer}
\end{align*}


	
\end{document}