\documentclass{article}
\usepackage[utf8]{inputenc}
\usepackage[T1]{fontenc}
\usepackage{indentfirst, hyperref, gensymb, amsmath, amsthm, wasysym, amsfonts, mathtools, braket, amssymb}
\hypersetup{colorlinks,allcolors=blue,linktoc=all}

\title{Chapter 1 Section 2 Exercise Solutions}
\author{Samuel Lair}
\date{September 2022}

\begin{document}

\maketitle
\tableofcontents

\pagebreak

\section{Exercise 1}
\[
	\neg a \implies \neg e
\]

\pagebreak

\section{Exercise 2}
\[
	m \implies e \lor p
\]

\pagebreak

\section{Exercise 3}
\[
	g \implies r \land \neg m \land \neg b
\]

\pagebreak

\section{Exercise 4}
\[
	\neg s \implies (d \implies w)
\]

\pagebreak

\section{Exercise 5}
\[
	e \implies a \land (b \lor p) \land r
\]

\pagebreak

\section{Exercise 6}
\[
	u \implies (b_{32} \land g_1 \land r_1 \land h_{16}) \lor (b_{64} \land g_2 \land r_2 \land h_{32})
\]

\pagebreak

\section{Exercise 7}
\subsection{(a)}
\[
	q \implies p
\]
\subsection{(b)}
\[
	q \land \neg p
\]
\subsection{(c)}
\[
	q \implies p
\]
\subsection{(d)}
\[
	\neg q \implies \neg p
\]

\pagebreak

\section{Exercise 8}

\subsection{(a)}
\[
	r \land \neg p
\]

\subsection{(b)}
\[
	r \land p \implies q
\]

\subsection{(c)}
\[
	\neg r \implies \neg q
\]
\subsection{(d)}
\[
	\neg p \land r \implies q
\]

\pagebreak

\section{Exercise 9}
Let
\begin{align*}
	p & \Coloneqq \text{"The system is in multiuser state."} \\
	q & \Coloneqq \text{"The system is operating normally."} \\
	r & \Coloneqq \text{"The kernel is functioning."}        \\
	s & \Coloneqq \text{"The system is in interrupt mode"}
\end{align*}
Then our system specifications can be expressed as the following system of logical expressions:
\begin{align}
	 & p \iff q \label{ex9eq1}          \\
	 & q \implies r \label{ex9eq2}      \\
	 & \neg r \lor s \label{ex9eq3}     \\
	 & \neg p \implies s \label{ex9eq4} \\
	 & \neg s \label{ex9eq5}
\end{align}
In order for \eqref{ex9eq5} to be true, $s$ must be false.  Since $s$ is false, $p$ must be true in order for \eqref{ex9eq4} to be true. Since $p$ is true, $q$ must be true in order for \eqref{ex9eq1} to be true. Since $q$ is true, $r$ must be true in order in order for \eqref{ex9eq2} to be true.  However, we must conclude that \eqref{ex9eq3} is false since $r$ is true and $s$ is false.

Therefore, there is no assignment of truth values such that all of our logical expressions are true. Hence, our system specifications are inconsistent.

\pagebreak

\section{Exercise 10}
Let
\begin{align*}
	p & \Coloneqq \text{"The system software is being upgraded."} \\
	q & \Coloneqq \text{"Users can access the file system."}      \\
	r & \Coloneqq \text{"Users can save new files"}
\end{align*}
Then our system specifications can be expressed as the following system of logical expressions:
\begin{align}
	p      & \implies \neg q \label{ex10eq1} \\
	q      & \implies r \label{ex10eq2}      \\
	\neg r & \implies \neg p \label{ex10eq3}
\end{align}
All of our logical expressions are true if we take $p = true$, $q = false$ and $r = true$. Hence, our system specifications are consistent.

\pagebreak

\section{Exercise 11}
Let
\begin{align*}
	p & \Coloneqq \text{"The router can send packets to the edge system."} \\
	q & \Coloneqq \text{"The router supports the new address space."}      \\
	r & \Coloneqq \text{"The latest software release is installed."}
\end{align*}
Then our system specifications can be expressed as the following system of logical expressions:
\begin{align}
	 & p \implies q \label{ex11eq1} \\
	 & q \implies r \label{ex11eq2} \\
	 & r \implies p \label{ex11eq3} \\
	 & \neg q \label{ex11eq4}
\end{align}
All of our logical expressions are true if we take $p = false$, $q = false$, and $r = false$. Hence, our system specifications are consistent.

\pagebreak

\section{Exercise 12}
Let
\begin{align*}
	p & \Coloneqq \text{"The file system is locked."}                       \\
	q & \Coloneqq \text{"New messages will be queued."}                     \\
	r & \Coloneqq \text{"The system is functioning normally."}              \\
	s & \Coloneqq \text{"New messages will be sent to the message buffer."}
\end{align*}
Then our system specifications can be expressed as the following system of logical expressions:
\begin{align}
	 & \neg p \implies q \label{ex12eq1} \\
	 & \neg p \iff r \label{ex12eq2}     \\
	 & \neg q \implies s \label{ex12eq3} \\
	 & \neg p \implies s \label{ex12eq4} \\
	 & \neg s \label{ex12eq5}
\end{align}
In order for \eqref{ex12eq5} to be true, s must be false. Since s is false, p must be true in order for \eqref{ex12eq4} to be true. Since s is false, q must be true in order for \eqref{ex12eq3} to be true. However, since p is false and q is false, we must conclude that \eqref{ex12eq1} is false.

All of our logical expressions are true if we take $p = true$, $q = true$, $r = false$, and $s = false$. Hence, our system specifications are consistent.

\pagebreak

\section{Exercise 13}
\subsection{(a)}
beaches AND New AND Jersey
\subsection{(b)}
(beaches AND Jersey) NOT New

\pagebreak

\section{Exercise 14}
\subsection{(a)}
hiking AND West AND Virginia
\subsection{(b)}
(hiking AND Virginia) NOT West

\pagebreak

\section{Exercise 15}
Ethiopian AND restaurant AND New AND (York OR Jersey)

\pagebreak

\section{Exercise 16}
(men AND (shoes or boots)) NOT work

\pagebreak

\section{Exercise 17}
\subsection{(a)}
The statement that "All of the inscriptions are false" is equivalent to the propositional expression:
\[
	\neg p_3 \land \neg p_1 \land \neg (\neg p_3) \equiv \neg p_1 \land \neg p_3 \land p_3 \equiv \neg p_1 \land F \equiv F
\]
Therefore, the Queen who never lies cannot make this statement.

\subsection{(b)}
The statement that "Exactly one of the inscriptions is true" is equivalent to the propositional expression:
\begin{align*}
	 & (p_3 \land \neg p_1 \land \neg (\neg p_3)) \lor (\neg p_3 \land p_1 \land \neg (\neg p_3)) \lor (\neg p_3 \land \neg p_1 \land \neg p_3) & \equiv \\
	 & (p_3 \land \neg p_1) \lor (\neg p_3 \land \neg p_1)
\end{align*}
The Queen who never lies could make this statement if the treasure is in either Trunk 3 or Trunk 2.

\subsection{(c)}
The statement that "Exactly two of the inscriptions are true" is equivalent to the propositional expression:
\begin{align*}
	 & (p_3 \land p_1 \land \neg (\neg p_3)) \lor (p_3 \land \neg p_1 \land \neg p_3) \lor (\neg p_3 \land p_1 \land \neg p3) & \equiv \\
	 & (p_3 \land p_1) \lor (\neg p_3 \land p_1)
\end{align*}
The Queen who never lies could make this statement. If we assume that the treasure cannot be in multiple trunks, we can conclude that the treasure is in Trunk 1.

\subsection{(d)}
The statement that "All three inscriptions are true" is equivalent to the propositional expression:
\begin{align*}
	 & p_3 \land p_1 \land \neg p_3 \equiv p_1 \land p_3 \land \neg p_3 \equiv p_1 \land F \equiv F
\end{align*}
The Queen who never lies cannot make this statement.

\pagebreak

\section{Exercise 18}
\subsection{(a)}
The statement that "All of the inscriptions are false" is equivalent to the propositional expression:

\[
	\neg (\neg p_1) \land \neg p_1 \land \neg p_2 \equiv p_1 \land \neg p_1 \land \neg p_2 \equiv F \land \neg p_2 \equiv F
\]
Therefore, the Queen who never lies cannot make this statement.

\subsection{(b)}
The statement that "Exactly one of the inscriptions is true" is equivalent to the propositional expression:
\begin{align*}
	 & (\neg p_1 \land \neg p_1 \land \neg p_2) \lor (\neg (\neg p_1) \land p_1 \land \neg p_2) \lor (\neg (\neg p_1) \land \neg p_1 \land p_2) & \equiv \\
	 & (\neg p_1 \land p_2) \lor (p_1 \land \neg p_2)
\end{align*}
The Queen who never lies could make this statement if the treasure is in either Trunk 2 or Trunk 1.

\subsection{(c)}
The statement that "Exactly two of the inscriptions are true" is equivalent to the propositional expression:
\begin{align*}
	 & (\neg p_1                                 & p_1 & \neg p_2) \lor (\neg p1 \land \neg p_1 \land p_2) \lor (\neg (\neg p1) \land p_1 \land p_2) & \equiv \\
	 & (\neg p_1 \land p_2) \lor (p_1 \land p_2)
\end{align*}
The Queen who never lies could make this statement. If we assume that the treasure cannot be in multiple trunks, we can conclude that the treasure is in Trunk 2.

\subsection{(d)}
The statement that "All three inscriptions are true" is equivalent to the propositional expression:
\[
	\neg p_1 \land p_1 \land p_2 \equiv F \land p_2 \equiv F
\]
Therefore, the Queen who never lies cannot make this statement.

\pagebreak

\section{Exercise 23}
Let
\begin{align*}
	 & p \Coloneqq \text{"A is a knight."} \\
	 & q \Coloneqq \text{"B is a knight."}
\end{align*}
Then A's statement is equivalent to the propositional expression:
\[
	R \Coloneqq \neg p \lor \neg q
\]

First, let us consider the case where A is a knave. Then $\neg p$ is true and A's statement is false. We could informally reason that the negation of A's statement is "Neither of us is knave" or $p \land q$. However, let us try a more formal approach by applying De Morgan's Law for OR:

\begin{align*}
	 & \neg R                    & \equiv \\
	 & \neg (\neg p \lor \neg q) & \equiv \\
	 & p \land q
\end{align*}
Therefore, we've contradicted our assumption that $\neg p$ is true.

Next, let us consider the case where A is a knight. Then $p$ is true and A's statement is true.  In order for $R$ to evaluate to true, $\neg q$ must be true. I.e. B is a knave.

Hence, we conclude that A is a knight and B is a knave.

\pagebreak

\section{Exercise 24}
The case where A is a knight is equivalent to the following system specifications:
\begin{align*}
	 & p                                      \\
	 & p \land q                              \\
	 & (\neg p \land q) \lor (p \land \neg q)
\end{align*}
These specifications are inconsistent.  In order for the first two expressions to be true, $p$ must be true and $q$ must be true. However, these truth value assignments result in the third expression evaluating to false.

The case where A is a knave is equivalent to the following system specifications:
\begin{align*}
	 & \neg p                                     \\
	 & \neg (p \land q) \equiv \neg p \lor \neg q \\
	 & (\neg p \land q) \lor (p \land \neg q)
\end{align*}
These specifications are all satisfied for $p = F$ and $q = T$.

Hence, we conclude that A is a knave and B is a knight.

\pagebreak

\section{Exercise 25}
The case where A is a knight is equivalent to the following system specifications:
\begin{align*}
	 & p             \\
	 & \neg p \lor q
\end{align*}
These specifications are all satisfied for $p = T$ and $q = T$.

The case where A is a knave is equivalent to the following system specifications:
\begin{align*}
	 & \neg p                                     \\
	 & \neg (\neg p \lor q) \equiv p \land \neg q
\end{align*}
These specifications are inconsistent. In order for the first expression to be true, $p$ must be false.  However, this truth value assignment results in the second expression evaluating to false regardless of the value for $q$.

Hence, we conclude that A is a knight and B is a knight.

\pagebreak

\section{Exercise 26}
Let $R$ be the logical expression equivalent to A's statement:
\[
	R \Coloneqq p
\]

Let $S$ be the logical expression equivalent to B's statement:
\[
	S \Coloneqq q
\]

This scenario is equivalent to the system specifications:
\begin{align*}
	 & (R \land p) \lor (\neg R \land \neg p) \equiv p \lor \neg p \equiv T \\
	 & (S \land q) \lor (\neg S \land \neg q) \equiv q \lor \neg q \equiv T
\end{align*}

Any truth value assignment will satisfy all of the specifications.

Hence, we conclude that we cannot draw any conclusions in this scenario. $A$ can be either a knight or a knave independent of the status of $B$. $B$ can be either a knight or a knave independent of the status of $A$.

\pagebreak

\section{Exercise 27}
Let $R$ be the logical expression equivalent to A's statement:
\[
	\neg p \land \neg q
\]

This scenario is equivalent to the logical expression:
\begin{align*}
	 & (R \land p) \lor (\neg R \land \neg p)                                       & \equiv \\
	 & (\neg p \land \neg q \land p) \lor (\neg (\neg p \land \neg q) \land \neg p) & \equiv \\
	 & (p \land \neg p \land \neg q) \lor ((p \lor q) \land \neg p)                 & \equiv \\
	 & (F \land \neg q) \lor ((\neg p \land p) \lor (\neg p \land q))               & \equiv \\
	 & F \lor (\neg p \land q)                                                      & \equiv \\
	 & \neg p \land q
\end{align*}

Hence, we conclude that A is a knave and B is a knight.

\end{document}

