\documentclass{article}
\usepackage[utf8]{inputenc}
\usepackage[T1]{fontenc}
\usepackage{indentfirst, hyperref, gensymb, amsmath, amsthm, wasysym, amsfonts, mathtools, braket, amssymb}
\hypersetup{colorlinks,allcolors=blue,linktoc=all}

\title{Chapter 1 Section 2 Exercise Solutions}
\author{Samuel Lair}
\date{September 2022}

\begin{document}

\maketitle
\tableofcontents

\pagebreak

\section{Exercise 1}
\[
	\neg a \implies \neg e
\]

\pagebreak

\section{Exercise 2}
\[
	m \implies e \lor p
\]

\pagebreak

\section{Exercise 3}
\[
	g \implies r \land \neg m \land \neg b
\]

\pagebreak

\section{Exercise 4}
\[
	\neg s \implies (d \implies w)
\]

\pagebreak

\section{Exercise 5}
\[
	e \implies a \land (b \lor p) \land r
\]

\pagebreak

\section{Exercise 6}
\[
	u \implies (b_{32} \land g_1 \land r_1 \land h_{16}) \lor (b_{64} \land g_2 \land r_2 \land h_{32})
\]

\pagebreak

\section{Exercise 7}
\subsection{(a)}
\[
	q \implies p
\]
\subsection{(b)}
\[
	q \land \neg p
\]
\subsection{(c)}
\[
	q \implies p
\]
\subsection{(d)}
\[
	\neg q \implies \neg p
\]

\pagebreak

\section{Exercise 8}

\subsection{(a)}
\[
	r \land \neg p
\]

\subsection{(b)}
\[
	r \land p \implies q
\]

\subsection{(c)}
\[
	\neg r \implies \neg q
\]
\subsection{(d)}
\[
	\neg p \land r \implies q
\]

\pagebreak

\section{Exercise 9}
Let
\begin{align*}
	p & \Coloneqq \text{"The system is in multiuser state."} \\
	q & \Coloneqq \text{"The system is operating normally."} \\
	r & \Coloneqq \text{"The kernel is functioning."}        \\
	s & \Coloneqq \text{"The system is in interrupt mode"}
\end{align*}
Then our system specifications can be expressed as the following system of logical expressions:
\begin{align}
	 & p \iff q \label{ex9eq1}          \\
	 & q \implies r \label{ex9eq2}      \\
	 & \neg r \lor s \label{ex9eq3}     \\
	 & \neg p \implies s \label{ex9eq4} \\
	 & \neg s \label{ex9eq5}
\end{align}
In order for \eqref{ex9eq5} to be true, $s$ must be false.  Since $s$ is false, $p$ must be true in order for \eqref{ex9eq4} to be true. Since $p$ is true, $q$ must be true in order for \eqref{ex9eq1} to be true. Since $q$ is true, $r$ must be true in order in order for \eqref{ex9eq2} to be true.  However, we must conclude that \eqref{ex9eq3} is false since $r$ is true and $s$ is false.

Therefore, there is no assignment of truth values such that all of our logical expressions are true. Hence, our system specifications are inconsistent.

\pagebreak

\section{Exercise 10}
Let
\begin{align*}
	p & \Coloneqq \text{"The system software is being upgraded."} \\
	q & \Coloneqq \text{"Users can access the file system."}      \\
	r & \Coloneqq \text{"Users can save new files"}
\end{align*}
Then our system specifications can be expressed as the following system of logical expressions:
\begin{align}
	p      & \implies \neg q \label{ex10eq1} \\
	q      & \implies r \label{ex10eq2}      \\
	\neg r & \implies \neg p \label{ex10eq3}
\end{align}
All of our logical expressions are true if we take $p = true$, $q = false$ and $r = true$. Hence, our system specifications are consistent.

\pagebreak

\section{Exercise 11}
Let
\begin{align*}
	p & \Coloneqq \text{"The router can send packets to the edge system."} \\
	q & \Coloneqq \text{"The router supports the new address space."}      \\
	r & \Coloneqq \text{"The latest software release is installed."}
\end{align*}
Then our system specifications can be expressed as the following system of logical expressions:
\begin{align}
	 & p \implies q \label{ex11eq1} \\
	 & q \implies r \label{ex11eq2} \\
	 & r \implies p \label{ex11eq3} \\
	 & \neg q \label{ex11eq4}
\end{align}
All of our logical expressions are true if we take $p = false$, $q = false$, and $r = false$. Hence, our system specifications are consistent.

\pagebreak

\section{Exercise 12}
Let
\begin{align*}
	p & \Coloneqq \text{"The file system is locked."}                       \\
	q & \Coloneqq \text{"New messages will be queued."}                     \\
	r & \Coloneqq \text{"The system is functioning normally."}              \\
	s & \Coloneqq \text{"New messages will be sent to the message buffer."}
\end{align*}
Then our system specifications can be expressed as the following system of logical expressions:
\begin{align}
	 & \neg p \implies q \label{ex12eq1} \\
	 & \neg p \iff r \label{ex12eq2}     \\
	 & \neg q \implies s \label{ex12eq3} \\
	 & \neg p \implies s \label{ex12eq4} \\
	 & \neg s \label{ex12eq5}
\end{align}
In order for \eqref{ex12eq5} to be true, s must be false. Since s is false, p must be true in order for \eqref{ex12eq4} to be true. Since s is false, q must be true in order for \eqref{ex12eq3} to be true. However, since p is false and q is false, we must conclude that \eqref{ex12eq1} is false.

All of our logical expressions are true if we take $p = true$, $q = true$, $r = false$, and $s = false$. Hence, our system specifications are consistent.

\end{document}